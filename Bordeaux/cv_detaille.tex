\partie{Formation et parcours professionnel}

\section{Parcours}
\label{sec:formation}
\vspace{5pt}

\cventry{2023 - 2024\hfill}{Postdoctorat}{}{}{}{}
\cvitem{Sujet}{\emph{Méthodes de résolution pour l'arithmétique de Presburger}}
\cvitem{Laboratoire}{IMDEA Software Institute, Madrid, Espagne}
\cvitem{Démarage}{Novembre 2023}
\cvitem{Encadrants}{M. Pierre Ganty (Associate Research Professor, IMDEA Software Institute)}
\cvitem{}{M. Alessio Mansutti (Associate Research Professor, IMDEA Software Institute)}
\vspace{10pt}


\cventry{2020 - 2023 \hfill}{Doctorat}{Bourse ministérielle}{}{}{}
\cvitem{Titre}{\emph{\mbox{A Polyhedral Framework for Reachability Problems in Petri Nets}}}
\cvitem{Spécialité}{Informatique et Télécommunications}
\cvitem{Laboratoire}{Laboratoire d'Analyse et d'Architecture des Systèmes (LAAS-CNRS)}
\cvitem{Établissement}{Institut National des Sciences Appliquées de Toulouse (INSA Toulouse)}
\cvitem{Période}{1er octobre 2020 -- 31 octobre 2023}
\cvitem{Soutenance}{4 décembre 2023}
\cvitem{Directeurs}{M. François Vernadat (PU, INSA Toulouse)}
\cvitem{}{M. Didier Le Botlan (MCF, INSA Toulouse)}
\cvitem{}{M. Silvano Dal Zilio (CR, CNRS)}
\cvitem{Rapporteurs}{Mme Laure Petrucci (PU, Univ. Sorbonne Paris Nord)}
\cvitem{}{M. Igor Walukiewicz (DR, CNRS)}
\cvitem{Examinateurs}{Mme Béatrice Bérard (Prof. émerite, Sorbonne Université)}
\cvitem{}{M. Fabrice Kordon (PU, Sorbonne Université)}
\cvitem{Président}{M. Loïc Hélouët (DR, INRIA)}
\cvitem{Manuscrit}{\url{https://theses.hal.science/tel-04458457}}
\cvitem{Transparents}{\url{https://nicolasamat.github.io/Slides_PhD.pdf}}
\vspace{10pt}

\cventry{2019 - 2020 \hfill}{Master of Science in Informatics at Grenoble (MoSIG)}{}{}{}{}
\cvitem{Université}{Université Grenoble Alpes}
\cvitem{Spécialité}{High-confidence Embedded and Cyberphysical Systems (HECS)}
\cvitem{Mention}{Très bien (major de promotion)}
\cvitem{Mémoire}{\emph{A new approach for the symbolic model checking of Petri nets}}
\cvitem{Soutenance}{23 juin 2020}
\cvitem{Directeurs}{M. Silvano Dal Zilio (CR, CNRS)}
\cvitem{}{M. Hubert Garavel (DR, INRIA)}
\cvitem{Rapporteur}{M. Yann Thierry-Mieg (MCF, Sorbonne Université)}
\cvitem{Examinateurs}{M. Akram Idani (MCF, Grenoble INP -- ENSIMAG)}
\cvitem{}{Mme Laurence Pierre (PU, Université Grenoble Alpes)}
\cvitem{Financement}{Bourse d'excellence du LabEx PERSYVAL-Lab}
\newpage

\cventry{2017 - 2020 \hfill}{Ingénieur en Informatique}{Grenoble INP -- ENSIMAG}{}{}{}
\cvitem{Spécialité}{Ingénierie des Systèmes d'Information (ISI)}
\cvitem{Mention}{Très bien (major de promotion)}
\vspace{10pt}

\cventry{2015 - 2017 \hfill}{Classes Préparatoires}{La Prépa des INP de Toulouse}{}{}{}


\section{Expériences professionnelles}

\cventry{2019 \hfill}{Stage assistant ingénieur}{ARM Ltd.}{Cambridge (UK)}{}{}
\cvitem{Contribution}{J'ai réalisé un ensemble de modifications du pilote des
GPU Mali d'Arm pour permettre son exécution sur User-Mode Linux (UML), un noyau
Linux compilé qui peut être exécuté dans l'espace utilisateur comme un simple
programme. J'ai également proposé un correctif du noyau Linux pour fournir sur
UML un accès direct à la mémoire (DMA) et une compatibilité avec devicetree
(structure de donnée décrivant les composants matériels).}
\cvitem{Encadrant}{M. Chris Diamand (Senior Software Engineer, ARM Ltd.)}
\cvitem{Durée}{3 mois}
\vspace{10pt}

\cventry{2019 \hfill}{Introduction à la recherche en laboratoire}{LIG}{Grenoble (France)}{}{}
\cvitem{Contribution}{J'ai réalisé une formalisation de la logique de séparation
à l'aide de l'assistant de preuve Isabelle/HOL ainsi que la preuve de résultats
de réécriture de formules issus d'un article intitulé \og The
Bernays-Schönfinkel-Ramsey Class of Separation Logic on Arbitrary Domains \fg.}
\cvitem{Encadrants}{M. Mnacho Echenim (PU, Grenoble INP -- ENSIMAG)}
\cvitem{}{M. Nicolas Peltier (CR, CNRS)}
\cvitem{Durée}{Une journée par semaine durant un semestre}
\vspace{10pt}

\cventry{2017 \hfill}{Stage en laboratoire}{IRIT}{Toulouse (France)}{}{}
\cvitem{Contribution}{J'ai réalisé des améliorations de sécurité dans XPIR, un
logiciel open source permettant à un utilisateur de télécharger de manière
secrète un élément d'une base de données (le serveur de la base de données sait
qu'il a envoyé un élément à l'utilisateur, mais ne sait pas lequel). Un tel
protocole est appelé Private Information Retrieval (PIR) et dans le cas de XPIR
ce dernier repose sur un chiffrement homomorphe.}
\cvitem{Encadrant}{M. Carlos Aguilar Melchor (MDC, Toulouse INP -- ENSEEIHT)}
\cvitem{Durée}{2 mois}


\section{Compétences techniques}

\cvitem{Programmation}{OCaml, Python, C, Java, Ada}
\cvitem{Solveurs}{z3, MiniZinc}
\cvitem{Model-checking}{Uppaal, LNT, Tina (Selt \& Muse)}
\cvitem{Asst. de preuve}{Isabelle/HOL}
\cvitem{Script}{Shell}
\cvitem{Hardware}{VHDL}


