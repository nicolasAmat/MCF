\partie{Projet d'enseignement}
\label{sec:projet_enseignement}
\vspace{10pt}

Durant ma thèse j'ai fait le choix d'effectuer des enseignements de \hl{tous
niveaux} (L1 à M2), dans des thématiques variées. Tout comme je valorise une
approche de la recherche qui combine des avancées théoriques avec des
implémentations concrètes, j'accorde une importance particulière à enseigner
l'informatique sous tous ses aspects, allant des \hl{fondements techniques} aux
\hl{notions plus théoriques}.\\

De ce constat, je pense être opérationnel pour enseigner sur les parcours
licence dans des UE en \hl{algorithmique} (notions que j'ai déjà enseignées à
l'INSA Toulouse) parmi les UE suivantes : algorithmique des structures de
données élémentaires, algorithmique des structures de données arborescentes et
algorithmique des graphes. L'UE \og \hl{programmation fonctionnelle} \fg de la
licence informatique est également commune aux vacations que j'ai
effectuées à l'INSA Toulouse. Fort de mon expérience en développement logiciel
(de par ma formation initiale à l'ENSIMAG et mon travail de recherche),
j’enseignerais également avec passion des UE comme celles de \hl{programmation
C} et de \hl{programmation système} (commune à la licence professionnelle). J'ai
pu par ailleurs consolider mon expérience en développement \og bas-niveau\ \fg
lors d'un stage chez ARM sur le kernel Linux.\\

Concernant les aspects plus \hl{théoriques} de l'informatique je me projette sur
plusieurs thématiques d'enseignement. Pour la formation licence, j’apprécierais
évoluer au sein l'UE \og \hl{modèles de la programmation et du calcul} \fg, en lien
direct avec mes travaux de recherche. L'UE \og \hl{logique et preuve} \fg me
semble également pertinente, étant un utilisateur expérimenté des solveurs SMT
(le model-checker que je développe est principalement basé sur ce type de
raisonnement).\\

En ce qui concerne les \hl{parcours de magistère}, mon travail de recherche me
challenge au quotidien sur la \hl{conception de techniques de d'outils de
vérification} et mon expérience d'enseignant à l'Université Paul Sabatier m'a
permis de consolider mon approche pédagogique sur \hl{l'utilisation des méthodes
formelles} dans la phase de conception de systèmes complexes et
sûrs (modélisation par des automates ou des réseaux de Petri, vérification de
propriétés à l'aide de model-checkers, etc.). Je pense donc pouvoir m'intégrer
aisément et être force de proposition sur les parcours \hl{vérification
logicielle} et \hl{algorithmes et modèles} (dans lesquels les UE sont
majoritairement communes) mais aussi sur le parcours \hl{conception formelle} (avec l'UE du même nom).\\

Tout comme pour les parcours de licence, j’apprécierais m'impliquer sur les UE
en algorithmique : \hl{applied algorithms} et \hl{distributed algorithms}. Mon
expertise en model-checking me conforte dans l'idée que je serais force de
proposition sur les UE \og \hl{software verification} \fg et \og
\hl{introduction à la vérification} \fg. 

Au vu des axes prioritaires de l'université, je pense m'inscrire dans plusieurs
d'entre eux. Je souhaite \hl{innover dans les approches pédagogiques}, en
particulier sur la création de projets stimulants pour les étudiants. Un exemple
concret serait une extension du projet que j'ai créé pour des étudiants de M2 et
des doctorants à l'ENAC, consistant à concevoir un model-checker de réseaux de
Petri basé sur des méthodes SMT. Un tel projet pourrait être étendu en une
compétition interne à l'UE inspirée par le Model Checking Contest. Les étudiants
(en groupes) soumettraient leur outil sur une plateforme dédiée afin de
l'évaluer automatiquement sur un ensemble de requêtes. Cette approche proche du
\hl{jeu} et la \hl{compétition} les inciterait à s'approprier la littérature et
innover dans le but d'améliorer la performance de leur outil, en fonction du
score et du classement retourné par la plateforme mise en place. Je suis
convaincu que la création de défis logiciels est un moteur dans la
compréhension des fondements théoriques.\\


Un autre axe intéressant dans la conception de projets pédagogiques est la
\hl{pluridisciplinarité}. Les enseignements que
j'ai effectués à l'Université Paul Sabatier étaient à mi-chemin entre la
vérification formelle et l'automatique (génération de commandes). Les étudiants
avaient à leur disposition des maquettes modélisant matériellement un système
réel (ascenseur, banc de tri d'objets, bras robotique et gare de triage). Je serais volontaire dans
le cadre de l'UE \og \hl{conception formelle} \fg pour élaborer une maquette
visant à sensibiliser les étudiants sur des problèmes de génération de commande
via l'utilisation de méthodes formelles sur des systèmes réels. Cette
réalisation serait envisageable grâce à la décharge d'enseignement des deux
premières années.\\

Pour conclure et sur un plan plus personnel, j'ai eu la chance d'avoir eu pour
formation le Master of Science in Informatics at Grenoble (MoSIG) où les cours
sont dispensés en \hl{anglais}, avec une majorité d'étudiants étrangers.
Conscient des bienfaits d'une telle formation je souhaiterais réaliser mes
enseignements de magistère en anglais et pouvoir participer à des enseignements
dans une \hl{université partenaire} comme celle de Ho Chi Minh.
