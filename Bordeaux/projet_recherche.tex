\partie{Projet de recherche}
\label{sec:projet_recherche}
\vspace{10pt}

\hl{\underline{Vérification de modèles d’un point de vue solveur}}
\vspace{10pt}
% Cette section débute par une introduction générale décrivant le problème
% théorique au centre de mon projet, ainsi que l'approche que je souhaite
% développer. La seconde partie est une description plus technique de mon projet
% de recherche et de mes objectifs. Le projet est suivi (Section~3) par une série
% de problèmes connexes, envisageables à court terme. Je termine la présentation
% de mon projet dans la Section~4 par une discussion sur les opportunités
% d'ouvertures, le type de modèles considérés et sur des problématiques de
% confiance dans les méthodes de vérification, démontrant le champ applicatif
% varié d'un tel projet. Le document se conclut par mon projet d'intégration dans
% l'équipe MTV du département M2F.

% \vspace{10pt}
% \section{Introduction et état de l'art}
% \vspace{10pt}

% Pour maîtriser la complexité toujours plus grande des \hl{systèmes critiques},
% il est nécessaire d'améliorer les méthodes et outils d'ingénierie des systèmes.
% La défaillance de tels systèmes peut avoir un impact négatif majeur sur la
% société, tant sur le plan financier qu'humain. Les \hl{méthodes formelles}
% offrent une réponse partielle à ce problème par l'utilisation d'une base
% mathématique solide, l'existence de techniques de vérification automatiques et
% la mise à disposition d'outils performants pour supporter ces techniques.\\

Dans mon projet de recherche je souhaite m'intéresser à la vérification par
modèles (ou \hl{{model-checking}}) qui est une technique de vérification
formelle définissant un ensemble de méthodes algorithmiques permettant de
déterminer si un \hln{modèle abstrait} satisfait une \hln{spécification
formelle}, généralement exprimée sous la forme d'une formule de logique
temporelle. Cette technique de vérification s'utilise aux \hl{différents stades de
développement des systèmes} (conception, architecture, \dots) et se base,
sommairement, sur la construction de l'ensemble des états dans lesquels peut se
trouver le système ou sur des techniques de raisonnement symboliques. Un axe
majeur d'amélioration de ces outils concerne le \hl{passage à l'échelle},
c'est-à-dire la capacité à appliquer ces outils sur des systèmes dont la taille
et la complexité rendent jusqu'ici l'analyse impossible.\\

Je souhaite aborder particulièrement le problème de la vérification de
\hl{propriétés d'accessibilité} pour des programmes et des systèmes concurrents.
La question est de savoir si un état donné ou une classe d'états, peut être
atteint par le modèle d'un tel système. L'accessibilité est un problème
\hln{fondamental et difficile}\footnote{Voir l'article de vulgarisation de la
revue Quanta :\\
\url{https://www.quantamagazine.org/an-easy-sounding-problem-yields-numbers-too-big-for-our-universe-20231204/}},
avec de nombreuses \hln{applications pratiques}. Elle joue évidemment un rôle
important dans la \hl{vérification formelle des systèmes concurrents}, par exemple
pour tester l'absence \og d'états dangereux\fg, tels que les blocages, ou pour
tester qu'un invariant est préservé. Ce problème est utilisé pour étudier divers
types de systèmes, tels que des systèmes
logiciels,
des protocoles et des systèmes distribués, des systèmes
biologiques, des
workflows, etc.\\

Le choix d'un formalisme particulier n'est pas nécessairement le facteur
dominant en model-checking. En effet, il est souvent possible de traduire ou
d'interpréter un modèle d'un formalisme à un autre. Dans mon projet je souhaite
m'intéresser aux modèles concurrents dont la sémantique (relation de transition)
peut être encodée dans \hl{l'arithmétique de Presburger} (théorie du premier
ordre des nombres entiers naturels munis de l'addition). Un exemple sont les
\hl{réseaux de Petri (RdP)}, que j'ai étudiés durant ma thèse de doctorat, ainsi
que leurs \hln{extensions}, qui sont l'un des premiers modèles proposés pour
l'étude de la \hl{théorie de la concurrence}. Notons que certains RdP peuvent
avoir un nombre non borné d'états accessibles, ce qui n'est pas la norme avec
d'autres formalismes. Dans ce cas, le problème de l'accessibilité fournit un cas
d'utilisation intéressant pour l'analyse des \hl{systèmes à états infinis}. Dans
mes travaux futurs, je souhaite donc continuer à développer des méthodes de
vérification adaptées aux réseaux non bornés.\\

Dans le cas général, le problème de l'accessibilité dans les RdP est l'un des
problèmes décidables les plus complexes puisqu'il est \hl{Ackermann-complet}. Une
conséquence pratique de cette \og complexité inhérente \fg qui fait consensus au
sein de la communauté Petri est que nous ne devrions pas nous attendre à trouver
un algorithme unique utilisable en pratique. Une meilleure stratégie consiste à
essayer d'améliorer les performances dans certains cas, par exemple en
développant de nouveaux outils ou des optimisations spécifiques à certains cas ;
ou bien essayer d'étendre la classe de problèmes que nous pouvons traiter. 
% Cette situation reflète un problème équivalent dans le
% domaine des solveurs SMT.
Ce constat est illustré par l'état de l'art actuel au Model Checking Contest
(MCC). En effet, les trois meilleurs outils des derniers
concours---\textsf{ITS-Tools},
\textsf{TAPAAL}, et \textsf{SMPT}
(que je développe)---s'appuient tous sur un portfolio d'approches et mélangent
différentes méthodes.\\

Historiquement les premiers model-checkers symboliques étaient basés sur les
\hl{diagrammes de décision binaires (BDD)}, une structure de données capable de
représenter efficacement la table de vérité des fonctions booléennes et donc
aussi des ensembles de vecteurs booléens. Cependant, ces techniques sont
\hln{limitées aux réseaux bornés} et, en pratique, ne sont pas à la hauteur des
\hl{approches basées SMT} (\emph{Satisfaisabilité Modulo Théories}).\\

Dans ce contexte, il est naturel de chercher à développer des techniques de
déduction automatique, du type solveurs SMT, qui permettent de décider des
propriétés reposant sur la théorie de l'arithmétique linéaire sur les entiers
(ou bien des structures telles que les vecteurs de bits, qui permettent de coder
des ensembles d'entiers bornés).
%%
En effet, les progrès réalisés par les \hln{solveurs SMT} ces dernières années
ont conduit à une \hln{avancée significative} des outils de \hln{vérification}.
Néanmoins, les solveurs restent parfois trop généraux et gagneraient à être
\hln{adaptés} et \hln{optimisés} pour des \hln{problèmes spécifiques}. Dans ce
contexte, j'envisage de développer des \hl{solveurs spécifiques} et des
procédures de décision qui prennent mieux en compte les \hln{formalismes
sous-jacents}. Ce projet est motivé par l'aspect central que tient l'utilisation de
l'arithmétique de Presburger dans l'analyse des propriétés des réseaux de Petri.
Ces résultats et outils pourraient avoir des applications dans plusieurs
domaines ; pour le \hln{model-checking} évidemment, mais aussi pour résoudre des
problèmes survenant dans la \hl{planification} et l'\hl{analyse
d'ordonnancement}.\\

C'est cette thématique de recherche, à la frontière de \hln{deux communautés} :
celle de la \hl{vérification formelle} et celle du \hl{raisonnement automatisé},
que je souhaite développer dans mes travaux de recherche et qui explique mon
choix de sujet de postdoctorat (méthodes de résolution pour l'arithmétique de
Presburger). Dans mon projet, je défends également une approche qui mêle
avancées théoriques et implantations pratiques ; que j'aime appeler des \og
théorèmes qui s'exécutent\fg.


% L'objectif de ce projet est de concevoir de
% nouvelles techniques efficaces de vérification automatique combinant des
% approches de type model-checking avec des techniques et résultats issus du
% domaine du raisonnement automatisé, en particulier de la \emph{Satisfaisabilité
% Modulo Théories} (SMT). 


% Un aspect essentiel lié à l'étude des réseaux de Petri consiste à
% s'appuyer largement sur les techniques de \hl{programmation linéaire en nombres
% entiers}, avec des concepts spécifiques tel que l'\emph{équation d'état} ou
% l'utilisation d'\emph{invariants structurels} par exemple. Ce point est central
% dans mon projet, situé à la frontière entre méthodes de résolution pour
% l'\hl{arithmétique de Presburger} et développement de techniques de
% model-checking.\\





% \begin{mdframed}
%   \textbf{\textbf{Axe de recherche 1:}} Développer des techniques de vérification pour les réseaux
%   de Petri, les modèles associés (tels que les VASS) et leurs extensions, en tirant parti de leur relation étroite avec
%   l'arithmétique de Presburger.
% \end{mdframed}


% \begin{mdframed}
%   \textbf{\textbf{Axe de recherche 2:}} S'attaquer au problème d'accessibilité
%   par le développement d'outils
%   concrets, dans le but de réduire l'écart entre décidabilité théorique et
%   faisabilité pratique.
% \end{mdframed}

% \begin{mdframed}
%     \textbf{\textbf{Axe de recherche 3:}} Approfondir les interactions entre
%     raisonnement automatisé et vérification formelle en mêlant avancées
%     théoriques et approche logicielle.
% \end{mdframed}


\vspace{10pt}
\section*{Intégration}
\vspace{10pt}

Je souhaite intégrer l'équipe MTV (\emph{Modelling and Technologies for
Verification}) du département M2F (\emph{Méthodes et Modèles Formels}).\\

Mon projet de recherche se base sur la relation étroite entre les réseaux de
Petri et l'arithmétique de Presburger, une relation qui a été explorée par
Jérôme Leroux, directeur de recherche dans l'équipe, et sur laquelle il a
proposé plusieurs résultats que j'utilise. Ces résultats théoriques sont souvent
le fruit de collaborations avec Wojciech Czerwiński, S\l{}awek Lasota, Ranko Lazic
et Filip Mazowiecki, tous chercheurs à l'Université de Varsovie. Un tel projet
de recherche s'insérerait donc dans cette relation privilégiée entre le LaBRI et
l'Université de Varsovie.\\

Ce projet mêle également abondamment recherche théorique et
développement logiciel. Une collaboration sur le développement d'un outil de
model-checking avec Frédéric Herbreteau et Gérald Point, qui maintiennent
l'outil TChecker, permettrait de mettre en œuvre les aspects pratiques de mon
projet de recherche, mais aussi d'envisager des extensions temporelles.\\

Sur les aspects plus théoriques, des interactions avec Igor
Walukiewicz et Grégoire Sutre sont envisageables, étant donné leurs expertises
sur des problèmes liés à l'utilisation de la logique, des automates et des
problèmes d'accessibilité, qui font également partie de mes centres d'intérêt.\\

Pour finir, des interactions avec les autres équipes du département M2F au sein
du groupe de travail \og preuve \fg seraient également un atout pour mon projet
de recherche, de par sa dualité vérification formelle et raisonnent automatisé,
mais aussi pour développer des aspects de certification et de preuve d'invariants.
