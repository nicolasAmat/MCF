\partie{Rayonnement scientifique et collaborations}
\label{sec:rayonnement}
\phantomsection\addcontentsline{toc}{chapter}{Rayonnement scientifique et collaborations}

\vspace{10pt}
\section{Distinctions}
\vspace{10pt}
L'ensemble des distinctions que j'ai obtenu sont :

\begin{itemize}
  \item \textbf{Médaille de bronze au Model Checking Contest 2023}.
  \begin{mdframed}
    
    Mon outil, \textsf{SMPT}, a remporté la médaille de bronze dans la catégorie
    \og accessibilité \fg du Model Checking Contest 2023, une compétition
    internationale d'outils de model-checking pour la vérification de systèmes
    concurrents.
  \end{mdframed}\smallbreak

  \item \textbf{Médaille de bronze et prix \og 100\% de confiance \fg au Model
  Checking Contest 2022}. 
  \begin{mdframed}
    Mon outil, \textsf{SMPT}, a remporté la médaille de bronze dans la catégorie
    \og accessibilité \fg du Model Checking Contest 2022. Il a également obtenu cette
    année-là le prix \og 100\% de confiance \fg, indiquant que l'outil n'a retourné aucun résultat erroné.  
  \end{mdframed}
  \smallbreak
  \item \textbf{Prix du meilleur teaser vidéo à la conférence Petri Nets 2021}
  \begin{mdframed}
    Pour le teaser de la présentation du papier \og On the Combination of Polyhedral
    Abstraction and SMT-based Model Checking for Petri nets \fg.
  \end{mdframed}\smallbreak
  \item \textbf{Bourse d'excellence du LaBex PERSYVAL-Lab}
  \begin{mdframed}
    Programme de bourses destiné à attirer des candidats exceptionnels en deuxième
    année d'un de ses masters liés aux disciplines d'un laboratoire du LaBex
    PERSYVAL-Lab.
  \end{mdframed}

\end{itemize}

\vspace{10pt}
\section{Présentations}
\vspace{10pt}

En plus des présentations de mes articles en conférence, j'ai eu l'opportunité de
présenter mes travaux à diverses occasions que je résume ci-dessous :
\vspace{10pt}

\subsection*{2023}
\vspace{5pt}

\begin{itemize}
  \item Automated Proof of Polyhedral Abstraction for Petri Nets,
  \textit{Séminaire du département M2F du LaBRI}, Bordeaux, France

  \item What is polyhedral reduction?... and how we use it to accelerate the
  verification of reachability problems for Petri nets, \textit{IMDEA Software
  Institute}, Madrid, Espagne. 
  
  \item Computing Linear Inductive Invariants for Petri Nets using Property
  Directed Reachability, \textit{GT AFSEC -- CT SED}, Paris, France.

  \item Property Directed Reachability for Generalized Petri Nets, \textit{IFSE:
  journées FAC}, Toulouse, France.
\end{itemize}

\newpage
\subsection*{2022}

\vspace{5pt}
\begin{itemize}
  \item What is Polyhedral Reduction? ... and how we use it to accelerate the
  verification of reachability problems, \textit{Séminaire de l'équipe MTV du
  LaBRI}, Bordeaux, France

  \item Computing Linear Inductive Invariants for Petri Nets using Property
  Directed Reachability, \textit{GT VERIF}, Bordeaux, France
\end{itemize}

\vspace{5pt}
\subsection*{2021}

\vspace{5pt}
\begin{itemize}
  \item On the Combination of Polyhedral Abstraction and SMT-Based Model
  Checking for Petri Nets, \textit{IFSE: journées FAC}, Toulouse, France.

  \item Une approche polyédrique pour la vérification SMT de réseaux de Petri,
  \textit{GT AFSEC -- GDR GPL}, Virtuel.
\end{itemize}

\vspace{10pt}
\section{Projets}
\vspace{10pt}
Je participe actuellement à deux projets européens :

\begin{itemize}
  \item \textbf{DECO} : Foundation of DEcentralized COncurrency. 
  \begin{mdframed}
    Septembre 2023 -- Août 2026
    \smallbreak
    Projet financé
    par le fonds européen de développement régional (FEDER).
  \end{mdframed}
  \smallbreak
  \item \textbf{PRODIGY} : Ensuring the security, scalability and functionality of digital provenance systems.
  \begin{mdframed}
    Décembre 2022 -- Novembre 2024
    \smallbreak
    Projet financé par NextGenerationEU.
  \end{mdframed}
\end{itemize}
\smallbreak

J'ai également pris part au développement de la boîte à outils
TINA\footnote{\url{https://projects.laas.fr/tina/index.php}}, initiée en 1983.

\vspace{10pt}
\section{Collaborations}
\vspace{10pt}

Durant ma thèse j'ai collaboré avec Hubert Garavel et Pierre Bouvier de l'équipe
CONVECS à INRIA Grenoble, sur le problème des places concurrentes. Cette
collaboration a mené la publication suivante :

\begin{itemize}
  \item[$\diamond$] \fullcite{amat_toolchain_2023}
\end{itemize}


