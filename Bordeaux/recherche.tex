\partie{Production scientifique}
\label{sec:recherche}


\vspace{10pt}
\section*{Description synthétique de mes travaux passés}
\vspace{10pt}

Mes travaux de recherche s'intéressent à la théorie et aux applications des
procédures de décision pour la vérification formelle. J'ai effectué ma thèse de
doctorat au LAAS-CNRS à Toulouse où je travaillais sur de nouvelles méthodes
pour exploiter des réductions de réseaux de Petri avec un model-checker basé sur
des méthodes SMT. Je travaille actuellement sur l'arithmétique de Presburger
dans le cadre d'un postdoctorat effectué à l'IMDEA Software Institute de Madrid

\vspace{10pt}
\subsection*{Thèse de doctorat}
\vspace{10pt}


Au cours de ma thèse de doctorat, j'ai proposé et étudié une méthode pour
contrôler l'explosion combinatoire lors de la vérification de problèmes
d'accessibilité sur les réseaux de Petri basée sur des réductions structurelles
appelées réductions polyédriques~\cite{amat_combination_2021,amat_polyhedral_2022}. L'approche est basée
sur une abstraction de l'espace d'états qui combine réductions structurelles et
contraintes arithmétiques sur le marquage des places.\\

La correction de cette méthode est basée sur une nouvelle notion d'équivalence
comportementale entre les réseaux. Combiné avec un model-checker basé sur des
méthodes SMT, je propose de transformer un problème d'accessibilité sur un
réseau de Petri en la vérification d'une propriété d'accessibilité sur une
version réduite de ce réseau~\cite{amat_combination_2021,amat_polyhedral_2022}. En exploitant un lien
avec une classe de réseaux de Petri qui ont un ensemble d'accessibilité
définissable par l'arithmétique de Presburger, j'ai également proposé
une procédure automatisée pour prouver qu'une telle abstraction est
correcte~\cite{amat_automated_2023}.\\

De plus, j'ai développé une structure de données, appelée Token Flow Graph
(TFG), qui capture la structure particulière des contraintes résultant des
réductions polyédriques~\cite{amat_accelerating_2021,amat_leveraging_2022}. J'ai exploité les TFGs
pour résoudre efficacement deux problèmes. Premièrement, pour éliminer les
quantificateurs, qui apparaissent lors de notre transformation, dans la formule
à vérifier sur le réseau réduit~\cite{amat_project_2024}. Deuxièmement, pour le
calcul de la relation de concurrence d'un réseau, c'est-à-dire énumérer toutes
les paires de places qui peuvent être marquées simultanément dans un marquage
accessible. Ce travail a conduit à une collaboration avec l'équipe CONVECS de
INRIA Grenoble~\cite{amat_toolchain_2023}.\\

J'ai appliqué mon approche à plusieurs procédures de model-checking symboliques.
Un des résultats de ce travail est la définition d'une nouvelle procédure de
semi-décision pour la vérification de propriétés d'accessibilité sur les réseaux
de Petri, basée sur la méthode Property Directed Reachability
(PDR)~\cite{amat_property_2022}. Une particularité de cette méthode PDR réside
dans sa capacité à générer des certificats de verdict dans l'arithmétique de
Presburger qui peuvent être vérifiés à l'aide d'un solveur SMT externe.\\

Je valorise une approche de la recherche qui combine des avancées théoriques avec
des implémentations concrètes. J'ai mis en œuvre mes résultats et mes algorithmes
dans quatre outils open source : \textsf{SMPT} pour vérifier des propriétés
d'accessibilité~\cite{amat_smpt_2023} ; \textsf{Kong} pour accélérer le calcul de places concurrentes~\cite{amat_kong_2022}
; \textsf{Octant} pour l'élimination de quantificateurs ; et \textsf{Reductron} pour
prouver automatiquement la correction de réductions polyédriques. J'ai étudié
leur efficacité dans le cadre d'évaluations expérimentales approfondies, à la
fois pour des réseaux bornés et non bornés, en utilisant les modèles et formules
fournies par le Model Checking Contest
(MCC)\footnote{\url{https://mcc.lip6.fr}}, une compétition annuelle et
internationale pour les outils de model-checking. Cela m'a conduit à participer
à la catégorie ``accessibilité'' des trois dernières éditions du MCC. Mon outil,
\textsf{SMPT}, a obtenu la médaille de bronze lors des deux dernières éditions
(2022 et 2023).

\vspace{10pt}
\subsection*{Postdoctorat}
\vspace{10pt}

Dans le cadre de mon postdoctorat réalisé à l'IMDEA Software Institute de
Madrid, je m'intéresse à la résolution de fragments spécifiques de
l'arithmétique de Presburger. D'un point de vue théorique, l'arithmétique de
Presburger est connue pour être décidable. Cependant, tout comme pour le
problème d'accessibilité sa complexité théorique (quelque part entre 2EXPTIME et
3EXPTIME) fait de sa résolution un véritable défi. Une conséquence pratique de
cette \og complexité inhérente \fg est qu'une stratégie intéressante consiste à
essayer d'améliorer les performances sur certains fragments particuliers pour
traiter de grandes formules provenant de problèmes de vérification concrets.\\

Si nous nous limitons au fragment sans quantificateur (également connu sous le
nom de Quantifier-Free Linear Integer Arithmetic, ou QF-LIA en abrégé), les
progrès récents des solveurs SMT permettent désormais de développer des
techniques de vérification efficaces pour le model-checking, l'interprétation
abstraite, etc. Cependant, ce fragment n'est pas assez expressif pour de
nombreux problèmes concrets, tels que le raisonnement sur différentes bases, la
recherche de points fixes, ou la vérification d'interpolants, qui nécessitent
souvent l'ajout de quantificateurs existentiels et peuvent être exprimés comme un
problème d'inclusion entre formules.\\

Dans ce contexte, je développe de nouvelles méthodes de vérification pour
des fragments spécifiques. Un premier fragment intéressant, comme mentionné
précédemment, est l'inclusion de formules existentielles. Fondamentalement,
une formule existentielle implique-t-elle (ou est-elle équivalente à) une autre
? Plus abstraitement, ce fragment peut être considéré comme un problème
d'inclusion de langage.  J'ai récemment développé une approche qui repose sur
un cadre théorique pour l'inclusion de deux langages, basé sur la notion de
quasi-ordre, qui réduit le problème à un nombre fini de questions
d'appartenance. 
% Nos premiers résultats montrent que cette
% approche innovante est plus efficace dans la pratique que les approches
% \og géométriques\fg  plus générales.\\

\vspace{10pt}
\subsection*{Autres travaux}
\vspace{10pt}

Durant mes études à l'ENSIMAG, j'ai travaillé avec de deux mes enseignants sur
la formalisation de la logique de séparation dans l'assistant de preuve
Isabelle/HOL, et la preuve de résultats de réécriture de formules. Ce travail
est disponible
librement~\footnote{\url{https://github.com/nicolasAmat/Separation-Logic-Formalization}}.

\vspace{10pt}
\section*{Liste des publications}
\label{sec:publications}
\vspace{10pt}

Mes travaux ont mené à 10 publications internationales (dont je suis le
co-auteur principal) : 3 articles de revue (Fundamenta Informaticae, STTT et
ToPNoC) et 7 articles de conférence parmi des conférences généralistes de rang A
tel que TACAS et FM (taux d'acceptation inférieurs à 30\%), la conférence VMCAI,
la conférence de référence sur les réseaux de Petri (PETRI NETS) et celle de
référence en model-checking (SPIN). À noter que deux autres articles de revue
sont en cours de publication (un accepté et un autre soumis).
\medbreak

Ci-après une liste de mes publications classées par type de communication et
par ordre chronologique. Le symbole ($\bigstar$) indique que j'ai choisi la
publication comme publication de référence (5 au total).
% Le symbole ($\bigstar$) indique que j'ai choisi la
% publication comme publication de référence (5 au total).

\vspace{10pt}
\subsection*{Revues internationales avec comité de lecture}
\vspace{10pt}

\begin{enumerate}
  \item ($\bigstar$) \cite{amat_polyhedral_2022} \fullcite{amat_polyhedral_2022} 
  \begin{mdframed}
    Cet article de revue introduit le cade théorique des réductions polyédriques
    appliquées à la vérification de propriétés d'accessibilité. J'y propose une
    nouvelle notion d'équivalence comportementale entre les réseaux, nommée
    \emph{équivalence polyédrique}, et je montre comment la combiner avec un
    model-checker basé sur des méthodes SMT. Ce travail a lancé le développement
    de mon outil \textsf{SMPT}, dans lequel j'ai appliqué mon approche à
    plusieurs procédures de model-checking symbolique. Les résultats
    expérimentaux ont montré que l'approche fonctionne bien sur le jeu de
    données du Model Checking Contest (reconnu par ma communauté), même
    lorsqu'une quantité modérée de réductions s'applique.
  \end{mdframed}
  \smallbreak
  % \textcolor{gray}{\textbf{Contribution :} Dans cet article de journal j'ai
  % défini une nouvelle méthode pour tirer parti des réductions de réseaux en
  % combinaison avec un model checker basé sur des méthodes SMT. L'approche
  % consiste à transformer un problème d'accessibilité sur un réseau de Petri en
  % la vérification d'une propriété d'accessibilité sur une version réduite de ce
  % réseau. Cette méthode repose sur une nouvelle abstraction de l'espace d'états
  % basée sur des systèmes de contraintes, appelée reduction polyédrique. Nous
  % prouvons la correction de cette méthode en utilisant une nouvelle notion
  % d'équivalence entre les réseaux. Notre approche a été mise en œuvre dans un
  % outil, appelé \textsf{SMPT}, qui fournit deux procédures principales : Bounded
  % Model Checking (BMC) et Property Directed Reachability (PDR). Nous avons testé
  % \textsf{SMPT} sur une large collection de problèmes utilisés au Model Checking
  % Contest. Nos résultats expérimentaux montrent que notre approche fonctionne
  % bien, même lorsque nous n'avons qu'une quantité modérée de réductions.}
	
  \item \cite{amat_leveraging_2022} \fullcite{amat_leveraging_2022} 
  % \smallbreak
  
  % \textcolor{gray}{\textbf{Contributions :} Dans cet article de revue je
	% propose une nouvelle structure de données, appelée Token Flow Graph (TFG), qui
	% capture la structure particulière des contraintes apparaissant dans les
	% réductions polyédriques. Nous utilisons les TFGs pour résoudre efficacement
	% deux problèmes d'accessibilité : d'abord pour vérifier l'accessibilité d'un
	% marquage donné et ensuite pour calculer la relation de concurrence d'un
	% réseau, c'est-à-dire énumérer toutes les paires de places qui peuvent être
	% marqués simultanément dans un marquage accessible. Nos algorithmes sont
	% implantés dans un outil, appelé \textsf{Kong}, que nous évaluons sur une
	% large collection de modèles utilisés lors du Model Checking Contest. Nos
	% expérimentations montrent que l'approche fonctionne bien, même lorsqu'une quantité
	% modérée de réductions s'applique.}
  \smallbreak
	\item ($\bigstar$) \cite{amat_toolchain_2023} \fullcite{amat_toolchain_2023}
  \begin{mdframed}
    Cet article publié dans la revue ToPNoC (Transactions on Petri Nets and Other
  Models of Concurrency) est le fruit de ma collaboration avec l’équipe CONVECS de
  INRIA Grenoble, sur le problème des places concurrentes. On y présente notre
  approche, combinant l'outil \caesar, qui fait partie de la boîte à outils
  \textsf{CADP}, avec les Token
  Flow Graphs qui sont une structure de données capturant la structure
  particulière des contraintes résultant de réductions polyédriques. Cette
  collaboration m’a également amené à développer l’outil
  \textsf{Kong}.
  \end{mdframed}
\end{enumerate}


\vspace{10pt}
\subsection*{Conférences internationales avec comité de lecture}
\vspace{10pt}
\begin{enumerate}
  \setcounter{enumi}{3}
  \item  \cite{amat_combination_2021} \fullcite{amat_combination_2021}
  \smallbreak
  \emph{Article sélectionné parmi les meilleures contributions pour être étendu en journal.}
  \smallbreak
  \item \cite{amat_accelerating_2021} \fullcite{amat_accelerating_2021}
  \smallbreak
  \emph{Article sélectionné parmi les meilleurs contributions pour être étendu en journal.}
  \smallbreak
  \item ($\bigstar$) \cite{amat_property_2022} \fullcite{amat_property_2022}
  \begin{mdframed}
    Dans cet article j'ai proposé un nouvel algorithme, qui est une procédure de
  semi-décision pour le problème d'accessibilité basée sur la méthode Property
  Directed Reachability (PDR), parfois aussi appelée IC3. Une caractéristique
  distinctive de cette extension de PDR aux réseaux de Petri est sa capacité à
  générer un \emph{certificat d'invariance}, sous la forme d'un invariant
  inductif de Presburger, lorsque nous constatons qu'une propriété est vérifiée
  par tous les états accessibles. J'ai choisi de présenter cette contribution
  car, à ma connaissance, elle est la seule méthode à fournir de tels
  certificats de manière effective dans le cas général (bien que l'existence
  théorique de tels certificats fût prouvée par Leroux en 2010).
  \end{mdframed}
  \smallbreak
  % \textcolor{gray}{\textbf{Contributions :} Dans cet article je propose une
  % procédure de semi-décision pour vérifier des propriétés d'accessibilité dans
  % les réseaux de Petri, basée sur la méthode PDR (Property Directed
  % Reachability). Nous définissons trois versions différentes, qui sont capables
  % de traiter des problèmes d'une difficulté croissante. J'ai implanté ces
  % méthodes dans mon model checker SMPT et nous donnons des preuves empiriques
  % que notre approche peut traiter problèmes difficiles ou impossibles à vérifier
  % avec les outils de l'état de l'art.}
  % \smallbreak
  \item \cite{amat_kong_2022} \fullcite{amat_kong_2022}
  \smallbreak
  \emph{Article sélectionné parmi les meilleurs contributions pour être étendu en journal.}
  \smallbreak
  \item ($\bigstar$) \cite{amat_smpt_2023} \fullcite{amat_smpt_2023}
  \begin{mdframed}
    Dans cet article je présente mon outil \textsf{SMPT} (\emph{Satisfiability
    Modulo Petri Nets}), sous licence libre (GPLv3). \textsf{SMPT} est un
    model-checker pour la vérification de propriétés d'accessibilité dans les
    réseaux de Petri, basé sur des méthodes SMT. Il a la particularité de tirer
    parti des réductions polyédriques. Cet outil a une importance particulière car
    il est le fondement de mes travaux expérimentaux de thèse et représente plus de
    $4\,500$ lignes de code (et $5\, 000$ lignes de commentaires et documentation). 
  \end{mdframed}
  \smallbreak
  \item \cite{amat_automated_2023} \fullcite{amat_automated_2023}
  \smallbreak
  \emph{Article sélectionné parmi les meilleurs contributions pour être étendu en journal.}
  \smallbreak
  % \textcolor{gray}{\textbf{Contributions :} Dans cet article je propose une
  % procédure automatisée pour prouver la corrections d'équivalences
  % polyédriques pour les réseaux de Petri. Mon approche repose sur un encodage
  % en un ensemble de formules SMT dont la satisfaction implique que l'équivalence
  % est correcte. La difficulté, dans ce contexte, provient du fait que nous devons
  % traiter des systèmes à états infinis. Pour la complétude, nous exploitons un
  % lien avec une classe de réseaux de Petri qui ont un ensemble d'accessibilité
  % définissables par l'arithmétique de Presburger. J'ai implanté
  % cette procédure dans un outil appelé \textsf{Reductron}.}
  % \smallbreak
  \item ($\bigstar$) \cite{amat_project_2024} \fullcite{amat_project_2024}
  \begin{mdframed}
    Cet article est représentatif de mon projet de recherche, mêlant vérification
  formelle et raisonnement automatisé. J'y ai proposé une méthode pour vérifier
  des propriétés d'accessibilité dans les réseaux de Petri en tirant parti de
  réduction polyédrique de manière transparente, comme une étape de prétraitement
  pour n'importe quel outil de model-checking. Mon approche est basée sur une
  nouvelle procédure qui peut projeter une propriété, d'un réseau de Petri
  initial, en une formule équivalente qui ne fait référence qu'à sa version
  réduite. Cette projection est définie comme une procédure d'élimination de
  quantificateurs pour l'arithmétique de Presburger, adaptée au fragment
  spécifique résultant de réductions. Les résultats expérimentaux ont montré
  que (1) sur notre fragment, mon outil de projection \textsf{Octant} est
  plus efficace que les outils généraux pour l'arithmétique de Presburger ; et (2)
  les réductions polyédriques sont efficaces sur un grand nombre de propriétés
  d'accessibilité, orthogonales aux optimisations existantes et que leurs
  avantages se combinent avec d'autres optimisations et model-checkers.
  \end{mdframed}
\end{enumerate}

\vspace{10pt}
% \vspace{1em}
\subsection*{Preprints}
\vspace{5pt}
\begin{enumerate}
  \setcounter{enumi}{10}
  \item \fullcite{preprint_1}\smallbreak
  \item \fullcite{preprint_2}
\end{enumerate}



\vspace{5pt}
\subsection*{Poster}
\vspace{5pt}
\begin{enumerate}
  \setcounter{enumi}{12}
  \item \fullcite{poster}
\end{enumerate}
\vspace{5pt}


\section*{Logiciels}
\vspace{10pt}
J'assure le développement et la maintenance de 4 outils open source (licence
GPLv3), en liens directs avec mes travaux de thèse, que je décris ci-dessous.
Pour chaque logiciel, la page GitHub correspondante contient le manuel
d'utilisation.
\vspace{10pt}

\begin{enumerate}
  \setcounter{enumi}{13}
  \item \textbf{SMPT: The Satisfiability Modulo Petri Nets Model Checker.}\\
  \url{https://github.com/nicolasAmat/SMPT}
  \begin{mdframed}
  Un model-checker basé sur des méthodes SMT pour les réseaux de Petri, axé sur
  les problèmes d'accessibilité, qui tire parti de réductions polyédriques.
  \textsf{SMPT} a participé aux trois dernières éditions du Model Checking
  Contest dans la catégorie \og accessibilité \fg et a obtenu la médaille de bronze
  en 2022 et 2023.
  \smallbreak
  \emph{Environ $4\ 500$ lignes de code et $5\, 000$ lignes de commentaire.}
  \end{mdframed}

  \item \textbf{Octant: The Reachability Formula Projector.}\\
  \url{https://github.com/nicolasAmat/Octant}
    \begin{mdframed}
  Un outil pour projeter des propriétés d'accessibilité d'un réseaux de Petri
  sur sa version réduite. Cet outil est utilisé dans le fonctionnement interne
  du model-checker \textsf{SMPT}.
  \smallbreak
  \emph{Environ $1\,500$ lignes de code.}
    \end{mdframed}
  
  \item \textbf{Kong: The Koncurrent Places Grinder.}\\
  \url{https://github.com/nicolasAmat/Kong}
  \begin{mdframed}
  Un outil pour accélérer le calcul de la relation de concurrence d'un réseau de
  Petri en utilisant des réductions polyédriques. Cet outil a été réalisé dans
  le cadre d'une collaboration avec l'équipe CONVECS de INRIA Grenoble.
  \smallbreak
  \emph{Environ $1\ 000$ lignes de code.}
  \end{mdframed}
  \newpage

  \item \textbf{Reductron: The Polyhedral Abstraction Prover.}\\
  \url{https://github.com/nicolasAmat/Reductron}
  \begin{mdframed}
  Un outil pour prouver automatiquement la correction d'équivalences
  polyédriques pour les réseaux de Petri.
  \smallbreak
  \emph{Environ $1\ 000$ lignes de code.}
  \end{mdframed}
\end{enumerate}
\vspace{10pt}



\section*{Science ouverte}
\vspace{10pt}
J'accorde une grande importance aux aspects \og science ouverte \fg dans mes travaux
de recherche. Cela s'est traduit par la développement de quatre outils open
source (décrits précédemment), la production de jeux de données, la mise à
disposition d'artefacts pour faciliter la reproduction de mes résultats et la
participation aux trois dernières éditions du Model Checking Contest, une
compétition internationale d'outils de model-checking pour la vérification de
systèmes concurrents. Ci-dessous je liste ces contributions.

\vspace{10pt}
\subsection*{Jeux de données}
\vspace{10pt}
J'ai contribué à la proposition de plusieurs modèles utilisés dans le benchmark du Model Checking Contest, en 
définissant trois modèles. J'ai également soumis une base de donnée de formules ajoutées au répertoire SMT-LIB
utilisé durant la compétition SMT-COMP.
\vspace{5pt}
\begin{enumerate}
  \setcounter{enumi}{17}
  \item \fullcite{amat_cryptominer_2023}\smallbreak
  \item \fullcite{amat_pgcd_2023}\smallbreak
  \item \fullcite{amat_murphy_2023}\smallbreak
  \item \fullcite{amat_qf-lia_2023}
\end{enumerate}

\vspace{10pt}
\subsection*{Artefacts}
\vspace{5pt}

J'ai fourni trois artefacts sur la plateforme \textsf{Zenodo}, pour les
conférences ayant une évaluation d'artefacts dans leur processus de soumission
(TACAS, FM et VMCAI). J'ai également produit un artefact couvrant l'ensemble de
mes travaux de thèse. L'objectif de ces artefacts est de permettre à la
communauté de reproduire mes résultats expérimentaux ainsi que de fournir un
environnement fonctionnel pour mes outils.
\vspace{1pt}
\begin{enumerate}
  \setcounter{enumi}{21}
  \item \fullcite{zenodo_artifact_2022}\smallbreak
  \item \fullcite{zenodo_artifact_2023}\smallbreak
  \item \fullcite{zenodo_artifact_2024}\smallbreak
  \item \fullcite{zenodo_artifact}
\end{enumerate}
\newpage


\subsection*{Compétition}
\vspace{10pt}

Ci-dessous je liste les références à mes trois participations au Model Checking Contest.
\vspace{10pt}
\begin{enumerate}
  \setcounter{enumi}{25}
  \item \textbf{2021.} \fullcite{kordon_complete_2021}\smallbreak
  \item \textbf{2022.} \fullcite{kordon_complete_2022}\smallbreak
  \item \textbf{2023.} \fullcite{kordon_complete_2023}
\end{enumerate}
