\partie{Enseignement, ressources pédagogiques et encadrement}
\label{sec:enseignements}
\phantomsection\addcontentsline{toc}{chapter}{Enseignement, ressources pédagogiques et encadrement}

\vspace{10pt}
\section{Enseignement}\label{sec:tab}
\vspace{10pt}

Dans cette section je décris les enseignements que j'ai effectués durant mes trois
années de thèse en tant que vacataire. Ces enseignements sont de tous niveaux
(L1 à M2) et ont été réalisés dans trois établissements différents (INSA
Toulouse, Université Toulouse III et ENAC).\\

{\small
\begin{tabular}{c @{\quad} p{13em} @{\qquad} c c c c}
\toprule
\, Niveau  & Intitulé &Formation& Nature &   \, Heures (1) \, & \, Eq. hTD (2)\,  \\
\midrule
\multicolumn{6}{l}{\textbf{2020-2021}}\\
\, L1 \,& \,Algorithmique en ADA \,&\, ing. INSA  \,&\,  TD / TP \,&\, 8,75 / 15 & 19,25\\
\midrule
\multicolumn{6}{l}{\textbf{2021-2022}}\\
\, M1 \,& \,SED, modélisation et analyse \,&\, univ. Toulouse III  \,&\,  TP \,&\, 32 & 21.33\\
\, M1 \,& \,Techniques de mises en œuvre pour les SED\,&\, univ. Toulouse III  \,&\,  TP \,&\, 30 & 20\\
\, M2 \,& \,Modèles temporels avancés \,&\, univ. Toulouse III  \,&\,  TD \,&\, 8 & 8\\
\midrule
\multicolumn{6}{l}{\textbf{2022-2023}}\\
\, L1 \,& \,Algorithmique en ADA \,&\, ing. INSA  \,&\,  TD / TP \,&\, 7,5 / 15 & 18\\
\, L3 \,& \,Expressions régulières \,&\, ing. INSA  \,&\,  TD \,&\, 5 & 5\\
\, M1 \,& \,Programmation fonctionnelle \,&\, ing. INSA  \,&\,  TP \,&\, 11 & 7,33\\
\, M2 \,& \,Modèles temporels avancés \,&\, univ. Toulouse III  \,&\,  TD \,&\, 8 & 8\\
\, M2 \,& \,SAT/SMT solving \,&\, ENAC \,&\,  TD \,&\, 6& 6\\
\bottomrule
\end{tabular}
}
\medbreak

(1) volume horaire en fonction de la nature, (2) volume horaire en équivalent TD.

\vspace{10pt}

\underline{\textbf{Total d'heures effectuées :}} 146,25 h (112,91 h équivalent TD).\\

Ci-dessous une description détaillée de mes enseignements par établissement.\\

\subsection{Institut National de Sciences Appliquées de Toulouse (INSA)}

\vspace{10pt}
\textbf{Algorithmique en ADA}.\medbreak 

Niveau : L1 -- 2020/2021 : 8,75 hTD et 15 hTP -- 2022/2023 : 7,5 hTD et 15 hTP.
% \medbreak 
\begin{mdframed}
  L'objectif de ce cours est d'enseigner les bases de l'algorithmique en utilisant
  le langage ADA comme support. En plus de consolider les acquis du semestre
  précédant (structures de contrôle), un point essentiel de ce cours est
  l'introduction de nouvelles structures de données, telles que les types
  énumérés, les tableaux, les matrices, ainsi que leur utilisation. Sur les années
  universitaires 2020/2021 et 2022/2023 j'avais en charge un groupe d'étudiants en
  TD et en TP. 
\end{mdframed}

\vspace{10pt}
\textbf{Expressions régulières}.\medbreak 

Niveau : L3 -- 2022/2023 : 5 hTD.
\begin{mdframed}
  L'objectif de ce cours est d'introduire les expressions régulières comme un
  outil pour l'ingénieur en informatique (par exemple, avec la commande
  \texttt{egrep}) ainsi que de rappeler les fondamentaux de l'utilisation des
  systèmes UNIX.
\end{mdframed}
\newpage
\vspace{10pt}
\textbf{Programmation fonctionnelle}.\medbreak

Niveau : M1 -- 2022/2023 : 11 hTP.
\begin{mdframed}
  Cours d'introduction à la programmation fonctionnelle avec le langage OCaml
  comme support. J'avais en charge un groupe d'étudiants durant leurs séances de
  TP. La difficulté de ce cours pour les étudiants réside à la fois dans le changement de
  paradigme (ne connaissant à ce stade que des langages impératifs ou orientés
  objets) ainsi que sur les aspects plus théoriques de typage (polymorphisme,
  ...).  
\end{mdframed}

\bigbreak

\subsection{Université Toulouse III --- Paul Sabatier}
\vspace{10pt}

\textbf{Systèmes à événements discrets, modélisation et analyse.}\medbreak

Niveau : M1 -- 2021/2022 : 32 hTP.
\begin{mdframed}
  Ce cours fournit les clés pour modéliser des systèmes à événements discrets avec
  des automates ou des réseaux de Petri. L'objectif final pour les étudiants est
  de synthétiser des commandes respectant un cahier des charges. Les TPs se
  découpent en maquettes (ascenseur, banc de tri d'objets, bras robotique et gare
  de triage). Pour chaque maquette le but est de modéliser le système donné,
  synthétiser une commande en se basant sur le cahier des charges et étudier les
  propriétés du modèle de commande obtenu (blocage, exclusion mutuelle, ...). Les
  élèves sont évalués sur un rapport et une soutenance orale.  
\end{mdframed}

\bigbreak


\textbf{Techniques de mises en œuvre pour les systèmes à événements discrets.}\medbreak

Niveau : M1 -- 2021/2022 : 30 hTP.
\begin{mdframed}
  Cet enseignement prolonge le cours précédant en mettant en œuvre des commandes
  de systèmes à événements discrets. Les TPs reprennent les bases de la
  modélisation et de la synthèse de commande. L'objectif ensuite est d'encadrer
  les étudiants sur un projet logiciel (en langage C), pour développer un
  programme qui prend en entrée une commande sous la forme d'un automate ou d'un
  réseau de Petri et qui renvoie une commande en langage VHDL, ST ou C, suivant la
  maquette cible. Les élèves sont évalués sur le code produit et un rapport.
\end{mdframed}
\bigbreak

\textbf{Modèles temporels avancés}. \medbreak

Niveau : M2 -- 2021/2022 : 8 hTD -- 2022/2023 : 8 hTD
\begin{mdframed}
  L'objectif de ce cours est de montrer comment des systèmes incluant des
  contraintes temporelles (par exemple, respect de durées minimales ou maximales
  des tâches à exécuter, durée d'attente entre des opérations élémentaires, temps
  d'acheminement d'éléments dans les systèmes, ...) peuvent être modélisés via des
  réseaux de Petri temporisés ou temporels. Les TDs que j'ai encadrés reprennent
  les fondamentaux des réseaux de Petri standards (graphe de couverture, graphe
  d'accessibilité, calcul d'invariants, modélisation, ...) et présentent ensuite
  un ensemble complet d’extensions temporisées et temporelles. Pour les réseaux de
  Petri temporels nous nous attardons particulièrement sur la construction du
  graphe des classes.  
\end{mdframed}

\newpage
\subsection{École National de l'Aviation Civile (ENAC)}
\vspace{10pt}

\textbf{SAT/SMT solving.}\medbreak

Niveau : M2 et doctorants  -- 2022/2023 : 6 hTD
\begin{mdframed}
  Ce cours vise à expliquer le fonctionnement des solveurs SAT et SMT. Dans ce
  cadre, j'ai développé un projet dont l'objectif est de montrer l'application des
  méthodes SMT dans la vérification de systèmes concurrents en développant un
  model-checker de réseaux de Petri pour les propriétés d'accessibilité. Le projet
  se découpe en deux phases. D'abord les étudiants répondent à des questions
  théoriques en écrivant des prédicats logiques. Ensuite, ils implémentent des
  fonctions dans un code à trous, où les structures des données, les parsers et la
  connexion avec le solveur sont déjà mis en place. Une spécificité de ce projet
  réside dans sa capacité à s'adapter au niveau des étudiants. En effet, les
  étudiants commencent par travailler avec des prédicats Booléens, puis ils peuvent
  ensuite basculer sur l'arithmétique de Presburger (par exemple pour traiter des
  systèmes à états infinis). Parmi les différentes méthodes de model-checking
  proposées, celles-ci sont également d'une difficulté incrémentale. Cette
  capacité d'adaptation provient d'une contrainte particulière, le public du cours
  est composé d'étudiants de M2 de l'ENAC (en master spécialisé), mais aussi de
  doctorants volontaires de l'école doctorale d'informatique et de mathématiques
  de Toulouse (EDMITT).
\end{mdframed}

% Ce projet pédagogique est librement accessible sur
% \textsf{GitHub}\footnote{\url{https://github.com/nicolasAmat/uSMPT}} et
% a été financé par ANITI (institut d'intelligence artificielle toulousain).

\vspace{10pt}
\section{Ressources pédagogiques}
\vspace{10pt}

Comme indiqué dans la description de mes enseignements, j'ai développé un projet
pédagogique, appelé \textsf{uSMPT}\footnote{\url{https://github.com/nicolasAmat/uSMPT}}. Ce projet pédagogique est librement
accessible sur
\textsf{GitHub} et a été
développée dans le cadre du volet formation de l'Institut Interdisciplinaire
d’Intelligence Artificielle (3IA) ANITI.

\begin{mdframed}
  \fullcite{amat_usmpt_nodate}
\end{mdframed}



\vspace{10pt}
\section{Encadrement}
\vspace{10pt}

Durant mes travaux de recherche j'ai co-encadré deux étudiants :

\begin{itemize}
  \item \textbf{Louis Chauvet} -- L3 à l'INSA Toulouse -- Du 7 juin au 30 juillet 2021 (3 mois)
  \smallbreak
  \textit{Implantation d'un algorithme de réduction de réseaux de Petri.}
  \begin{mdframed}
    L'objectif principal de ce stage était de formaliser un sous-ensemble
    expressif des réductions utilisées dans mes travaux et d'en proposer une
    implantation open source dans le langage
    Rust\tablefootnote{\url{https://github.com/Fomys/pnets}}. Ce stage a également mené
    la publication suivante :
    % \smallbreak
    \begin{itemize}
      \item[$\diamond$] \fullcite{amat_kong_2022}
    \end{itemize}
  \end{mdframed}
  \newpage
  \item \textbf{Sarah Moreau} -- L2 à la Prépa des INP de Toulouse -- Du 3 mai au 11 juin 2021 (6 semaines)
  
  \smallbreak
  \textit{Résolution de systèmes d'équations linéaires à variables entières ; extensions de l'algorithme de Contejean et Devie.}
  \begin{mdframed}
    L'objectif de ce stage était l'étude d'un algorithme de résolution de systèmes
    d'équations linéaires dû à Contejean et Devie. Cette approche, proposée à
    l'origine en 1994, permet de décider de l'existence d'une solution à valeurs
    entières, en exhibant un témoin représentant une solution minimale du système.
    Il s'agit d'une approche intéressante du point de vue algorithmique, reposant
    sur l'exploration de l'espace des solutions, qui est assez différente des
    méthodes \og plus classiques \fg, basées sur la diagonalisation du système ou sur
    une adaptation de la méthode du simplexe. Durant le stage, nous nous sommes
    intéressés à comment adapter cet algorithme dans le cas de systèmes
    d'équations particuliers, plus contraints, qui apparaissent dans mes travaux
    sur la vérification formelle de réseaux de Petri utilisant des
    approches dites SMT. 
  \end{mdframed}
\end{itemize}








