\partie{Projet d'enseignement}
\label{sec:projet_enseignement}
\phantomsection\addcontentsline{toc}{chapter}{Projet d'enseignement}
\vspace{10pt}

Durant ma thèse j'ai fait le choix d'effectuer des enseignements de \hl{tous
niveaux} (L1 à M2), dans des thématiques variées. Tout comme je valorise une
approche de la recherche qui combine des avancées théoriques avec des
implémentations concrètes, j'accorde une importance particulière à enseigner
l'informatique sous tous ses aspects, allant des \hl{fondements techniques} aux
\hl{notions plus théoriques}.\\

La formation dont j'ai pu bénéficier à l'ENSIMAG (dans la filière Ingénierie des
Systèmes d'Information) est une formation spécialisée en \hl{génie logiciel} et
\hl{systèmes}. Ce cursus m'a permis d'acquérir un large éventail de
\hl{compétences techniques} que j'ai su valoriser par la suite, par exemple lors
d'un stage de \hln{développement sur le noyau Linux} chez \hln{ARM} à Cambridge,
ou lors de mes travaux de thèse par la \hl{réalisation d'implantations
logicielles solides et reconnues} par la communauté de la vérification formelle.
Pour exemple, je suis le développeur principal de \hl{quatre outils
open-source}, développés avec \hl{différents paradigmes} de programmation :
orienté objet (en Python) et fonctionnel (en OCaml). Mon outil
principal, \textsf{SMPT}, est un model-checker de réseaux de Petri qui a obtenu
la médaille de bronze en 2022 et 2023 dans la catégorie \og accessibilité \fg du
Model Checking Contest. Ces travaux montrent que, bien que ma thématique de
recherche soit la vérification formelle, je suis un développeur confirmé avec
une passion pour le \hl{génie logiciel} et je pense être force de proposition
dans les parcours de licence et de master (SDL, IHM et CSA).\\

À partir de cette observation et de mon \hl{expérience} en tant qu'enseignant à
l'Université Paul Sabatier, mais aussi à l'INSA Toulouse, où j'ai enseigné les
\hl{principes de l'algorithmique} (en ADA), la \hl{programmation fonctionnelle}
(en OCaml) et l'utilisation des \hl{systèmes UNIX} (voir l'UE \og expressions
régulières \fg), je suis convaincu de ma capacité à enseigner les bases de
l'informatique en licence, notamment dans les domaines de l'algorithmique, du
développement logiciel et de l'architecture.\\

Concernant les aspects plus \hl{théoriques} de l'informatique je me projette sur
plusieurs thématiques d'enseignement. Parmi les matières que je pourrais
mentionner : l'UE \og \hl{ingénierie des systèmes et des modèles} \fg en lien
avec mon enseignement dans la formation EEA à l'université, l'UE \og
\hl{vérification et validation, analyse formelle} \fg en relation avec mes
travaux de recherche, l'UE \og \hl{algorithmique avancée} \fg en raison de mon
expérience dans le développement de procédures de décision, ou encore l'UE \og
\hl{parallélisme} \fg qui correspond à mon travail en tant que développeur.
Avant tout, je suis persuadé que je trouverais ma place dans des cours où il y a
un besoin et où je pourrais partager ma passion pour l'informatique.\\


Je souhaite également \hl{innover dans les approches pédagogiques}, en
particulier sur la création de projets stimulants pour les étudiants. Un exemple
concret serait une extension du projet que j'ai créé pour des étudiants de M2 et
des doctorants à l'ENAC, consistant à concevoir un model-checker de réseaux de
Petri basé sur des méthodes SMT. Un tel projet pourrait être étendu en une
compétition interne à l'UE inspirée par le Model Checking Contest. Les étudiants
(en groupes) soumettraient leur outil sur une plateforme dédiée afin de
l'évaluer automatiquement sur un ensemble de requêtes. Cette approche proche du
\hl{jeu} et de la \hl{compétition} les inciterait à s'approprier la littérature
et innover dans le but d'améliorer la performance de leur outil, en fonction du
score et du classement retourné par la plateforme mise en place. Je suis
convaincu que la création de défis logiciels est un moteur dans l'apprentissage
de l'informatique. Cette réalisation serait envisageable grâce à la décharge
d'enseignement des deux premières années.\\\\


Pour finir, un autre axe qui m'importe dans la conception de projets
pédagogiques est la \hl{pluridisciplinarité}. Les enseignements que j'ai
effectués à \hl{l'Université Paul Sabatier} (formation EEA) étaient à mi-chemin entre
la vérification formelle et l'automatique (génération de commandes). Les
étudiants avaient à leur disposition des maquettes modélisant matériellement un
système réel (ascenseur, banc de tri d'objets, bras robotique et gare de
triage). Cette approche pluridisciplinaire que je souhaiterais développer fait,
selon moi, tout son sens dans le master \hl{Computer Science for Aeroposace (CSA)}.\\